\section{Advanced Topics}

\begin{frame}
\frametitle{Building SimpleITK}
\end{frame}

\begin{frame}
\frametitle{SimpleITK architecture}
\end{frame}

\subsection{Code Philosophy}
\begin{frame}{Code Philosophy}
\fontsize{36pt}{36pt}\selectfont
\center
\begin{center}
Code Philosophy
\end{center}
\end{frame}

\begin{frame}[fragile]
\frametitle{Filter Class Overview (C++)}
\lstcpp
\begin{lstlisting}
class SmoothingRecursiveGaussianImageFilter : (* \label{Declaration} *)
    public ImageFilter {
  typedef SmoothingRecursiveGaussianImageFilter Self; (* \label{Self} *)

  /** Default Constructor that takes no arguments
      and initializes default parameters */
  SmoothingRecursiveGaussianImageFilter(); (* \label{Constructor} *)

\end{lstlisting}
\begin{itemize}
  \item In line \ref{Declaration}, we declare a subclass of ImageFilter
  \item Line \ref{Self} creates a special typedef for use later
  \item The default constructor is line \ref{Constructor} (never any parameters)
\end{itemize}
\end{frame}


\begin{frame}[fragile]
\frametitle{Filter Class Overview (C++) Continued}
\lstcpp
\begin{lstlisting}
  /** Define the pixels types supported by this filter */
  typedef BasicPixelIDTypeList  PixelIDTypeList; (* \label{PixelIDTypeList} *)

\end{lstlisting}
\begin{itemize}
  \item Notice $PixelIDTypeList$ in line \ref{PixelIDTypeList}
  \item Used to instantiate ITK filters
  \item Determines valid input image types
  \item $BasicPixelIDTypeList$ expands to:
  \begin{itemize}
    \item $int8\_t$, $uint8\_t$
    \item $int16\_t$, $uint16\_t$
    \item $int32\_t$, $uint32\_t$
    \item $float$, $double$
  \end{itemize}
\end{itemize}
\end{frame}


\begin{frame}[fragile]
\frametitle{Filter Class Overview (C++) Continued}
\lstcpp
\begin{lstlisting}
  Self& SetSigma ( double t ) { ... return *this; }
  double GetSigma() { return this->m_Sigma; }

  Self& SetNormalizeAcrossScale ( bool t ) { ... }
  Self& NormalizeAcrossScaleOn() { ... }
  Self& NormalizeAcrossScaleOff() { ... }

  bool GetNormalizeAcrossScale() { ... }
\end{lstlisting}
\begin{itemize}
  \item Get/Set parameters
  \item Set methods always return $Self\&$ (more later)
  \item Generally, a direct mapping to ITK
  \item Boolean parameters generate $On$ and $Off$ methods
\end{itemize}
\end{frame}


\begin{frame}[fragile]
\frametitle{Filter Class Overview (C++) Continued}
\lstcpp
\begin{lstlisting}
  /** Name of this class */
  std::string GetName() const { ... }

  /** Print ourselves out */
  std::string ToString() const;
\end{lstlisting}
\begin{itemize}
  \item Return the name and description of the filter
\end{itemize}
\end{frame}


\begin{frame}[fragile]
\frametitle{Filter Class Overview (C++) Continued}
\lstcpp
\begin{lstlisting}
  /** Execute the filter on the input image */
  Image Execute ( const Image & );

  /** Execute the filter with parameters */
  Image Execute ( const Image &,
    double inSigma,
    bool inNormalizeAcrossScale );
};  /* End of class SmoothingRecursiveGaussian */

Image SmoothingRecursiveGaussian ( const Image& , (* \label{Function} *)
  double inSigma = 1.0,
  bool inNormalizeAcrossScale = false );

\end{lstlisting}
\begin{itemize}
  \item Run the filter on an image and return the result
  \item Notice extra function (line \ref{Function}), adds flexibility
  \item Drop $ImageFilter$ from class name to get function name
\end{itemize}

\end{frame}

\begin{frame}{Questions?}
\fontsize{36pt}{36pt}\selectfont
\center
\begin{center}
Questions?
\end{center}
\end{frame}



\begin{frame}
\frametitle{Using ITK with SimpleITK}
\end{frame}

\begin{frame}
\frametitle{Interfacing with OpenCV (or other libraries)}
\end{frame}

\begin{frame}
\frametitle{Extending SimpleITK}
JSON and friends
\end{frame}
